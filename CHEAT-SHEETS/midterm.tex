\chapter{Midterm Cheatsheet}

\subsection{Factors Affecting Interest Rates}
\begin{itemize}
    \item Inflation - Higher inflation leads to higher interest rates
    \item Credit (default) Risk
          \begin{itemize}
              \item The risk that the borrower will not pay back the loan
              \item The higher the default risk, the higher the interest rate
          \end{itemize}
    \item Liquidity Risk
          \begin{itemize}
              \item Risk associated with bieng able to access the invested funds during the investment period
              \item The higher the liquidity risk, the higher the interest rate
          \end{itemize}
    \item Maturity Risk
          \begin{itemize}
              \item The risk that the value of the investment will decrease as a result of changes in interest rates
              \item The longer the maturity the longer the investment payoff, which also increases credit and liquidity risk, leading to higher interest rates
          \end{itemize}
\end{itemize}

\begin{claim}
    [Effective Interest Rate]
    \begin{equation}
        i_e = (1 + \frac{r}{m})^m -1
    \end{equation}
\end{claim}


\begin{corollary}
    [Continuous Compounding Effective Interest Rate]
    The limit of the effective interest rate as the number of compounding periods approaches infinity is the continuous compounding effective interest rate.
    \begin{align}
        i_e & = \lim_{m \to \infty} (1 + \frac{r}{m})^m - 1 = e^r - 1 \\
        r   & = \lim_{m \to \infty} ((i_e + 1) - 1)m = \ln(1 + i_e)
    \end{align}
\end{corollary}

\begin{theorem}
    [Future Interest Rates]
    \begin{align*}
        (1 + r_{1 \to x})^{\frac{x}{12}} & = (1 + r_{1 \to k})^{\frac{k}{12}} \cdot (1 + r_{k \to x})^{\frac{x-k}{12}}
    \end{align*}
    \begin{itemize}
        \item $r_{1 \to x}$ is the interest rate from time 1 to time $x$
        \item $r_{1 \to k}$ is the interest rate from time 1 to time $k$
        \item $r_{k \to x}$ is the interest rate from time $k$ to time $x$
    \end{itemize}
\end{theorem}

\begin{corollary}
    [Compound Amount Factor]
    \[
        (F/P, i, N) = (1+i)^N
    \]
\end{corollary}

\begin{definition}
    [Present Value of an Annuity]
    The present value of an annuity is equal to a perpetuity starting at the first period minus a perpetuity of the same size starting at the $N+1$ period.
    \begin{align}
        \boxed{(P/A, i, N) = \frac{1}{i}( 1 - \frac{1}{(1+i)^{N}})}
    \end{align}
    \textit{Note: The second annuity, like all annuities, is also worth $A/i$, but because it starts at the $N+1$ period, we discount it.}
\end{definition}


\begin{definition}
    [Present Value of an Arithmetic Growth Factor]
    The arithmetic gradient is the sum of $N$ annuities, each increasing by a constant amount $G$. Remember that arithmetic gradients begin at period 1.
    \begin{align*}
        \boxed{(P/G, i, N) = \frac{1}{i^2}\left(1 - \frac{1+iN}{(1 + i)^N}\right)}
    \end{align*}
    \textit{Note: To get the total arithmetic gradient present value, add the growth factor to the initial annuity such that: $P = A(P/A, i, N) + G(P/G, i, N)$}
\end{definition}

\begin{definition}
    [Present Value of a Geometric Gradient]
    Looking at the figure \ref{geom-cash-flow}, we can model the present value of a geometric gradient, keeping in mind to discount the cash-flows at the end of each period.
    \begin{align*}
        \boxed{(P/G, i, g, N) =\begin{aligned}
                                                  & \frac{1 - (\frac{1 + g}{1 + i})^N}{i - g}                                     \\
                                       \text{or } & \frac{(P/A, i^0, N)}{1 + g} \quad \text{where } i^0 = \frac{1 + i}{1 + g} - 1
                                   \end{aligned}}
    \end{align*}
\end{definition}

\begin{definition}
    [Loan to Value Ratio]
    The \textbf{loan to value ratio} is the ratio of the principle to the value of the house. The loan to value ratio is calculated as follows:
    \[
        \text{LTV} = \frac{\text{Principle}}{\text{House Value}}
    \]
\end{definition}


\begin{definition}
    [Amortization Period]
    The \textbf{amortization period} is the amount of time it takes to pay off the mortgage. The amortization period is typically 25 years.
\end{definition}


\begin{definition}
    [Mortgage Term]
    The \textbf{mortgage term} is the amount of time the mortgage rate is fixed. After which the mortgage rate is renegotiated.
    \\ The mortgage term is typically 5 years.
\end{definition}
\begin{definition}
    [Volatility]
    Volatility is a statistical measure of the dispersion of returns for a given security or market index.
    Volatility is the standard deviation of the returns.
    \begin{equation}
        \sigma = \sqrt{Var(\overrightarrow{R_i})} = \sqrt{\frac{1}{n-1} \sum_{i=1}^{n} (r_i - \bar{r})^2}
    \end{equation}
    where:
    \begin{itemize}
        \item $r_i$ is the return at time $i$.
        \item $\bar{r}$ is the average return.
        \item $n$ is the number of returns.
    \end{itemize}
\end{definition}

\begin{definition}
    [Covariance]
    The covariance between two assets is a measure of how the two assets move together. If the covariance is positive, the assets move together, i.e. when one asset increases, the other asset also increases.
    \begin{equation}
        Cov(\overrightarrow{R_i},\overrightarrow{R_j}) = \frac{1}{n-1} \sum_{i=1}^{n} (r_{i} - \bar{r_i})(r_{j} - \bar{r_j})
    \end{equation}
\end{definition}


\begin{definition}
    Assume we can proportionally invest in $x_i$ amount in the $i^{th}$ asset. We define:
    \begin{equation}
        \overrightarrow{X} = \begin{bmatrix}
            x_1    \\
            x_2    \\
            \vdots \\
            x_n
        \end{bmatrix}
    \end{equation}
    And the covariance matrix of the returns of the $n$ assets is given by:
    \begin{equation}
        \Sigma = \begin{bmatrix}
            \sigma_{11} & \sigma_{12} & \cdots & \sigma_{1n} \\
            \sigma_{21} & \sigma_{22} & \cdots & \sigma_{2n} \\
            \vdots      & \vdots      & \ddots & \vdots      \\
            \sigma_{n1} & \sigma_{n2} & \cdots & \sigma_{nn}
        \end{bmatrix}
    \end{equation}
    Where:
    \begin{itemize}
        \item $\sigma_{ij} = \sigma_{ji} = Cov(\overrightarrow{R_i},\overrightarrow{R_j})  $ is the covariance between the $i^{th}$ and $j^{th}$ asset.
        \item $\sigma_{ii} = Var(\overrightarrow{R_i})$ is the variance of the $i^{th}$ asset.
    \end{itemize}
\end{definition}


\begin{definition}
    [Portfolio Expected Return]
    The expected return of a portfolio is given by:
    \begin{equation}
        \mathbb{E}[\overrightarrow{R_p}] = x_1 \mathbb{E}[\overrightarrow{R_1}] + x_2 \mathbb{E}[\overrightarrow{R_2}] + \cdots + x_n \mathbb{E}[\overrightarrow{R_n}] = \sum_{i=1}^{n} x_i \mathbb{E}[\overrightarrow{R_i}]
    \end{equation}
    Where:
    \begin{itemize}
        \item $\mathbb{E}[\overrightarrow{R_i}]$ is the expected return of the $i^{th}$ asset.
        \item $\mathbb{E}[\overrightarrow{R_p}]$ is the expected return of the portfolio.
        \item $\overrightarrow{R_i}$ is the historical return of the $i^{th}$ asset.
        \item $x_i$ is the proportion of the $i^{th}$ asset in the portfolio.
    \end{itemize}
    Intuitively, the expected return of each asset is multiplied by how much of the asset is in the portfolio.
\end{definition}



\begin{definition}
    [Portfolio Volatility]
    The volatility of a portfolio is given by:
    \begin{equation}
        \sigma_p = \sqrt{\overrightarrow{X}^T \times \Sigma \times \overrightarrow{X}}
    \end{equation}
\end{definition}

\begin{theorem}
    [CAPM Assumption]
    The Capital Asset Pricing Model (CAPM) assumes that investors are rational and risk-averse, in addition, it is just a model with numerous assumptions.\\
    \textbf{Assumptions:}
    \begin{itemize}
        \item Correlation and volatility of and between assets are fixed and constant forever
        \item All investors aim to maximize economic utility (i.e. make as much money as possible, regardless of any other considerations)
        \item[] \begin{itemize}
                  \item Although usually true, we see irrational behavior in the market (e.g. driven by fear)
              \end{itemize}
        \item All investors are rational and risk-averse
        \item All investors have access to the same information at the same time
        \item Investors have accurate conception of possible returns, i.e. the probability beliefs of investors match the true distribution of returns.
        \item There are no taxes or transaction costs
        \item[] \begin{itemize}
                  \item Often they are negligible, so they are ignored
              \end{itemize}
        \item All investors are price takers, i.e. their actions do not influence prices
        \item Any investor can lend and borrow an unlimited amount at the risk free rate of interest
        \item All securities can be divided into parcel of any size
    \end{itemize}
\end{theorem}


\begin{definition}
    [Sharpe Ratio]
    To prove that this provides the largest ratio of excess return to risk, we realize that the slope of the tangent line is given by:
    \begin{align*}
        m & = \frac{y_2 - y_1}{x_2 - x_1}                \\
        m & = \frac{\mathbb{E}[R_p] - r_f}{\sigma_p - 0}
    \end{align*}
    This is the Sharpe Ratio, we want to maximize it, and it is the slope of the tangent line.\\
    Higher the slope, higher the sharpe ratio, higher the excess return for the same risk.
\end{definition}


\begin{proposition}
    [Leveraging]
    If we invest $50\%$ in the risk-free asset and $50\%$ in the market portfolio, we expect a return that is the average of the two returns, and a volatility that is the average of the two volatilities.\\
    \begin{equation}
        \mathbb{E}[R_{p}] = w\cdot \mathbb{E}[R_{m}] + (1-w)\cdot r_f
    \end{equation}

    \begin{equation}
        \sigma_{p} = \sqrt{w^2 \sigma_{m}^2 + (1-w)^2 \cdot \sigma_{f}}, \quad \sigma_{f} = 0
    \end{equation}
    Where $w$ is the weight of the market portfolio.\\
    This is called leveraging down, because we are investing in a less risky portfolio than the market portfolio.\\
\end{proposition}
\begin{definition}
    [Systematic Risk]
    Systematic risk is the risk that is inherent to the entire market or market segment. It is also known as undiversifiable risk or market risk $\beta$.\\
    E.g. If the economy is in a recession, all stocks will likely decrease in value.\\
\end{definition}

\begin{definition}
    [Idiosyncratic Risk]
    Idiosyncratic risk is the risk that is specific to a particular company or industry. It can be mitigated through diversification.\\
    E.g. If a company's CEO is caught embezzling money, the company's stock will likely decrease in value.\\
\end{definition}



\begin{theorem}
    [Expected Return of an asset]
    The expected return of an asset is given by:
    \begin{equation}
        \mathbb{E}[R_i] = r_f + \beta_i(\mathbb{E}[R_m] - r_f)
    \end{equation}
    Intuitively, it says that the expected return of an asset is the risk-free rate plus the amount that an increase in the market portfolio would increase the return of the asset.\\
    The systematic risk of the asset is defined as:
    \begin{equation}
        \beta_i = \frac{Cov(R_i,R_m)}{Var(R_m)} = \frac{\sigma_{i,m}}{\sigma_{m}^2} = \frac{\rho_{i,m} \sigma_i}{\sigma_{m}}
    \end{equation}
    Plotting $\mathbb{E}[R_i]$ vs. $\beta_i$ gives the Security Market Line (SML).\\
\end{theorem}

\begin{definition}
    [Expected Value]
    The expected value of a project is the weighted average of the possible outcomes of the project, where the weights are the probabilities of the outcomes.\\
    \begin{equation}
        \mathbb{E}[P] = \sum_{i=1}^{n} P_i \cdot p_i
    \end{equation}
    Where:
    \begin{itemize}
        \item $P_i$ is the possible outcome of the project.
        \item $p_i$ is the probability of the outcome.
    \end{itemize}
\end{definition}

\begin{definition}
    [Expected Future Payoff/Return]

    The expected future payoff of the project is given by the expected value of the project, over the present value of the project:

    \begin{equation}
        \mathbb{E}[R] = \frac{\mathbb{E}[P]}{P_0} -1 = \frac{\sum_{i=1}^{n} P_i \cdot p_i}{P_0} -1
    \end{equation}

\end{definition}
