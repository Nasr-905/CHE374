\chapterimage{./Images/head2.jpg} % Chapter heading image
\chapter{Time Value of Money }
\section{Present, Future Value, and Interest Rates}
\begin{theorem}
    [Engineering Project Life Cycle]
    \begin{itemize}
        \item[]
        \item Conceptulization
        \item Detailed Economic Analysis
        \item Decign
        \item Build
        \item Utilize/produce
        \item Phase out/dispose
    \end{itemize}
\end{theorem}

\begin{remark}
    We invest in projects to gain benefits  (e.g. economic, societial).
\end{remark}

\begin{theorem}
    [Tiume Value of Money]
    \begin{itemize}
        \item[]
        \item Bow
        \item Borrowing costs the borrower
        \item Lending money creates value for the lender
    \end{itemize}
\end{theorem}

\subsection{Present and Future Value}
\begin{vocabulary}
    \begin{itemize}
        \item[]
        \item $P$ is the present value (Principal)
        \item $F_N$ is the future value at time $N$
        \item $i$ is the interest rate
        \item $I$ is the interest amount
    \end{itemize}
\end{vocabulary}
\begin{theorem}
    [Future Value]
    \begin{equation}
        F = P(1+i)
    \end{equation}

\end{theorem}
\subsection{Factors Affecting Interest Rates}
\begin{itemize}
    \item Inflation - Higher inflation leads to higher interest rates
    \item Credit (default) Risk
          \begin{itemize}
              \item The risk that the borrower will not pay back the loan
              \item The higher the default risk, the higher the interest rate
          \end{itemize}
    \item Liquidity Risk
          \begin{itemize}
              \item Risk associated with bieng able to access the invested funds during the investment period
              \item The higher the liquidity risk, the higher the interest rate
          \end{itemize}
    \item Maturity Risk
          \begin{itemize}
              \item The risk that the value of the investment will decrease as a result of changes in interest rates
              \item The longer the maturity the longer the investment payoff, which also increases credit and liquidity risk, leading to higher interest rates
          \end{itemize}
\end{itemize}

\begin{example}
    [Quick Quiz]
    \begin{itemize}
        \item[]
        \item What would you expect the impact to be on the interest rate associated with a large new mining project under the following scenarios:
              \begin{itemize}
                  \item The expected time to complete the project construction increases,
                  \item The estimated amount of ore available to be mined increases?
              \end{itemize}
        \item How would the current value of the project be impacted under the above scenarios?
    \end{itemize}
    \begin{itemize}
        \item Maturing risk increases, time horizon increases, interest rate increases, current value decreases.
              \begin{itemize}
                  \item Could also be a liquidity risk because the money is tied up in the project and won't be available for longer.
                  \item Could also be credit risk because the project may not be completed and the mine may default on the loan.
                  \item We also expect that as the time horizon and the interest rate increases, the present value of the project decreases.
              \end{itemize}
        \item Future value increases, value of the project increases, credit risk decreases. So interest rate decreases.
    \end{itemize}
\end{example}

\section{Simple and Compound Interest}

\begin{definition}
    [Simple Interest]
    \textit{Simple interest} is calculated on the principal amount only.
\end{definition}

\begin{theorem}
    [Simple Interest]
    \begin{equation}
        F_N = P(1 + iN)
    \end{equation}
\end{theorem}

\begin{definition}
    [Compound Interest]
    \textit{Compound interest} is calculated on the principal amount and the interest that has been added to the principal.
\end{definition}

\begin{theorem}
    [Compound Interest]
    \begin{equation}
        F_N = P(1 + i)^N
    \end{equation}
\end{theorem}
\begin{example}
    \begin{itemize}
        \item If you borrow S100 for 3 years at 10\% annual interest (simple interest), how much interest will you pay, and how much will you owe after 3 years?
    \end{itemize}
    \begin{align*}
        I & = P \cdot i \cdot N = 100 \cdot 0.1 \cdot 3 = 30 \\
        F & = P + I = 100 + 30 = 130
    \end{align*}
\end{example}

\begin{example}
    \begin{itemize}
        \item If you borrow S100 for 3 years at 10\% annual interest (compound interest), how much interest will you pay, and how much will you owe after 3 years?
    \end{itemize}
    \begin{align*}
        I & = P \cdot (1 + i)^N - P = 100 \cdot (1 + 0.1)^3 - 100 = 100 \cdot 1.331 - 100 = 33.1 \\
        F & = P + I = 100 + 33.1 = 133.1
    \end{align*}
\end{example}

\section{Subperiod Compounding and Effective Interest Rates}

\begin{vocabulary}
    \begin{itemize}
        \item[]
        \item $r$ is the nominal interest rate (usually annual)
        \item $m$ is the number of compounding periods per year
        \item $i_s =\frac{r}{m}$ is the interest rate per subperiod
    \end{itemize}
\end{vocabulary}

\begin{theorem}
    [Subperiod Compounding]
    \begin{equation}
        F_N = P(1 + \frac{r}{m})^{mN}
    \end{equation}
\end{theorem}

\begin{definition}
    [Effective Interest Rate]
    The \textit{effective interest rate}, $i_e$, is the interest rate that would be required for one compounding period to achieve the same future value as the nominal interest rate with multiple compounding periods.
\end{definition}
\begin{claim}
    \begin{equation}
        i_e = (1 + \frac{r}{m})^m -1
    \end{equation}
\end{claim}


\begin{remark}
    [When Do We Use Effective Interest Rates vs. Periodic Interest Rates?]
    \begin{itemize}
        \item Use the nominal interest rate when calculating the future value of an investment with multiple compounding periods.
        \item Use the effective interest rate when comparing investments with different compounding periods.
    \end{itemize}
\end{remark}

\begin{proof}
    Say $F = P(1 + \frac{r}{m})^{mN}$ is our subperiod compounding interest rate. We wan to find the effective interest rate that would give us the same future value as $F$.
    \begin{align}
        F           & = P(1 + i_e)^N = P(1 + \frac{r}{m})^{mN} \\
        (1 + i_e)^N & = (1 + \frac{r}{m})^{mN}                 \\
    \end{align}
\end{proof}

\begin{corollary}
    [Continuous Compounding Effective Interest Rate]
    The limit of the effective interest rate as the number of compounding periods approaches infinity is the continuous compounding effective interest rate.
    \begin{align}
        i_e & = \lim_{m \to \infty} (1 + \frac{r}{m})^m - 1 = e^r - 1 \\
        r   & = \lim_{m \to \infty} ((i_e + 1) - 1)m = \ln(1 + i_e)
    \end{align}
\end{corollary}

\section{Problem Set Notes}
\begin{vocabulary}
    [Maturity]
    The time at which the principal and interest are due.
\end{vocabulary}
\begin{vocabulary}
    [Yield rate]
    The interest rate (as a measure of return on investment) that equates the present value of the cash flows to the initial investment.
\end{vocabulary}