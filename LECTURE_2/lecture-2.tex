\chapter{Cash-flow Analysis}


\section{Cash-flow Factors}

\begin{theorem}
    [Factor Notation]
    \begin{itemize}
        \item $(X/Y, i, N)$ - the factor that converts one kind of cash-flow to another
              \begin{itemize}
                  \item $X$ and $Y$ are one of the following:
                        \begin{itemize}
                            \item $F$ - future value
                            \item $P$ - present value
                            \item $A$ - annual value
                            \item $G$ - gradient
                            \item $g$ - growth rate
                        \end{itemize}
                  \item So, to convert from a present value $P$ to a future value $F$ at interest rate $i$ for $N$ periods, we use the factor $(P/F, i, N)$, which is read as "P given F at i for $N$ periods"
                  \item For the geometric gradient, we use $P = (P/G, i, g, N)$
              \end{itemize}
        \item Some factors have specific names
              \begin{itemize}
                  \item $(F/P, i, N)$ - Compound amount factor
                  \item $(P/F, i, N)$ - Present worth factor
                  \item $(A/F, i, N)$ - Sinking fund factor
                  \item $(F/A, i, N)$ - Series compound amount factor
                  \item $(A/P, i, N)$ - Capital recovery factor
                  \item $(P/A, i, N)$ - Series present worth factor
              \end{itemize}
    \end{itemize}
\end{theorem}

\begin{corallary}
    [Compound Amount Factor]
    \[
        F = P(1+i)^N
    \]
    \[
        (F/P, i, N) = (1+i)^N
    \]
\end{corallary}

\begin{example}
    If you had \$2000 now and invested it at 10\% interest, how much would you have in 8 years? \\
    \textbf{Solution:}
    \begin{align}
        P & = 2000, i = 0.1, N = 8 \\
        F & = P(P/F, i, N)         \\
        F & = 2000(P/F, 10\%, 8)
        F & = 2000(1+0.1)^8        \\
        F & = 2000(2.143)          \\
        F & = 4287.20
    \end{align}
\end{example}

\begin{example}
    You invested \$1100 today, and in 8 years you expect to have \$2000. What is the interest rate? \\
    \textbf{Solution:}
    \begin{align}
        P     & = 1100, F = 2000, N = 8 \\
        F = P(P/F, i, N)                \\
        2000  & = 1100(P/F, i, 8)       \\
        2000  & = 1100(1+i)^8           \\
        1.818 & = (1+i)^8               \\
        i = 7.8\%
    \end{align}
\end{example}

\begin{example}
    Say you invest \$100 and expect to be paid 5\% interest annually. What is the annual value? \\
    \textbf{Solution:}
    \begin{align}
        P & = 100, i = 0.05, N = 1 \\
        A & = P(P/A, i, N)         \\
        A & = 100(i)               \\
        A & = 100(0.05)            \\
        A & = 5
    \end{align}
\end{example}

\chapter{Lecture Problems}

\begin{Problem}
    Consider Annuity that pays $\$10$ per month for 1 year. If interest is $1\%$ per month, compounded monthly. What is the equivalent continuous payment rate? \\
    \textbf{Solution:}
    \begin{align}
        P   & = \$10 (p/A, 1\%, 12) = \$112.56 \\
        P_0 & = A\int_0^1 e^{r_{cc}t} dt       \\
        P_0 & = A\int_0^1 e^{0.01t} dt         \\
        r = 11.94\% / \text{year}
    \end{align}
\end{Problem}

\begin{problem}
Determine the effective annual rate for an investment that earns $15\%$ per year based on
quarterly compounding for the first $4$ months, then earns $11\%$ per year based on continuous
compounding for another $5$ months.
\end{problem}